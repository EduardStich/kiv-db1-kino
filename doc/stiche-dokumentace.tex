\documentclass[a4paper, 11pt]{article}

% kódování
\usepackage[T1]{fontenc}
\usepackage[utf8x]{inputenc}
\usepackage[czech]{babel}

% font
\usepackage[condensed]{roboto}
\renewcommand{\familydefault}{\sfdefault}

% uspořádání textu
\usepackage[expansion=all]{microtype}
\usepackage{ragged2e}

% odsazení
\usepackage[margin=2.5cm]{geometry} 

% grafika
\usepackage{graphicx}
\usepackage{pdfpages} % pdf na jednu stránku
\usepackage{booktabs,csvsimple}
\usepackage{caption}
\usepackage{amsmath}

\graphicspath{{img/}{pdf/}}

% sql kód
\usepackage{listings}
\usepackage{xcolor}

\lstset{
  language=SQL,
  basicstyle=\ttfamily\small,
  keywordstyle=\color{blue}\bfseries,
  commentstyle=\color{gray},
  stringstyle=\color{orange},
  showstringspaces=false,
  breaklines=true,
  frame=single,
  tabsize=2,
  % česká diakritika v sql
  literate=
    {á}{{\'a}}1 {č}{{\v{c}}}1 {ď}{{\v{d}}}1 {é}{{\'e}}1 {ě}{{\v{e}}}1
    {í}{{\'i}}1 {ň}{{\v{n}}}1 {ó}{{\'o}}1 {ř}{{\v{r}}}1 {š}{{\v{s}}}1
    {ť}{{\v{t}}}1 {ú}{{\'u}}1 {ů}{{\r{u}}}1 {ý}{{\'y}}1 {ž}{{\v{z}}}1
    {Á}{{\'A}}1 {Č}{{\v{C}}}1 {Ď}{{\v{D}}}1 {É}{{\'E}}1 {Ě}{{\v{E}}}1
    {Í}{{\'I}}1 {Ň}{{\v{N}}}1 {Ó}{{\'O}}1 {Ř}{{\v{R}}}1 {Š}{{\v{S}}}1
    {Ť}{{\v{T}}}1 {Ú}{{\'U}}1 {Ů}{{\r{U}}}1 {Ý}{{\'Y}}1 {Ž}{{\v{Z}}}1
}

% ostatní
\author{Eduard Štich}
\date{15. 11. 2025}

\begin{document}

% --- titulní strana ---
\thispagestyle{empty}

\includegraphics[width=2in]{image1.png}
\vfill

\begin{center}
\section*{\LARGE{Kino}}
\subsection*{KIV/DB1 -- Semestrální práce}
\end{center}
\vfill

\begin{tabular}{@{}ll@{}}
student: & Eduard Štich \\
osobní číslo: & A25B0287P \\
email: & stiche@students.zcu.cz \\
datum: & 15. 11. 2025
\end{tabular}
\newpage

% --- diagram ---
\includepdf[width=\linewidth]{novy-diagram.pdf}

% --- rozobr ---
\section{Detailní popis a charakteristika zadání semestrální práce}

\subsection{Zdůvodnění výběru tématu}

Jako nosné téma databáze jsem zvolil kino, protože jeho provoz tvoří přehlednou síť vzájemně propojených entit (filmy, místa, promítání, vstupenky, osoby), v níž se přirozeně vyskytují všechny požadované konstrukce: vazba M:N, 3. NF a číselníkové tabulky. Díky častým návštěvám různých kin dobře znám některé provozní procesy, a tak dokáži zvolit vhodná testovací data. Toto téma mi zároveň umožní v rámci předmětu KIV/WEB snadno vytvořit webovou aplikaci.

\subsection{Vymezení rozsahu}

Databáze pokrývá kompletní řetězec ,,film $\rightarrow$ promítání $\rightarrow$ vstupenka $\rightarrow$ divák'' a související agendu se zaměstnanci; nezahrnuje však další provozní okruhy. Uchovává tedy filmy a jejich žánry, místa s kapacitami, časová pásma promítání, jednotlivé vstupenky včetně stavu (vydáno/rezervováno/zaplaceno) a osoby -- diváky i zaměstnance -- včetně jejich přihlašovacích údajů.

Evidence zahrnuje pracovní pozice, směny, dny a bonusy, avšak neřeší kompletní mzdovou a účetní agendu. Dále neobsahuje skladové hospodaření s občerstvením, dodavatelské faktury, marketingové kampaně ani věrnostní systém; ty jsou ponechány jako možná rozšíření v navazujících předmětech.

\subsection{Slovní charakteristika zpracovávaných dat}

Datový model pracuje s devíti základními entitami, které pokrývají kompletní provozní agendu kina.

\begin{list}{}{}
\item Žánr (\texttt{ZANR}) slouží jako číselníková tabulka obsahující název a popis žánru; každý film je přiřazen právě k jednomu žánru.
\item Film (\texttt{FILM}) uchovává název, stopáž v minutách, stručný popis a cizí klíč na žánr. Jedná se o hlavní informační jednotkou, která je dále vázána na konkrétní promítání.
\item Místo (\texttt{MISTO}) reprezentuje promítací sál nebo jiný prostor kina. Má unikátní název, celkovou kapacitu sedadel a současně maximální počet zaměstnanců, kteří se zde mohou teoreticky vyskytovat během směny.
\item Promítání (\texttt{PROMITANI}) spojuje film, sál a časový údaj -- slouce s cizími klíči vázané na film a místo, dále sloupec „od“ a „do“ který určuje čas začátku a konce, sloupec „datum“ jako kalendářní den; jedno promítání může mít libovolný počet prodaných vstupenek -- teoreticky až tolik, kolik činí kapacita místa.
\item Osoba (\texttt{OSOBA}) eviduje fyzické osoby s unikátním přihlašovací jménem, heslem (v realizované aplikaci pak hashované), kontaktním e-mailem a aktuálním kreditem v Kč. Příznak „soucast\_personalu“ rozlišuje běžného diváka od zaměstnance.
\item Vstupenka (\texttt{VSTUPENKA}) obsahuje cenu, příznak zaplaceno, číslo sedadla a cizí klíče na promítání a volitelně na osobu; každá vstupenka existuje v jediném exempláři a její stav určuje, zda je rezervace závazná.
\item Smena (\texttt{SMENA}) definuje časový úsek práce („od“ a „do“) spolu s případným bonusem na hodinu; na jednu směnu je navázána sada pracovních pozic.
\item Pracovni pozice (\texttt{PRACOVNI\_POZICE}) konkretizuje roli zaměstnance (např. pokladní, promítač, úklid), popis činnosti, základní mzdu a příznak „funguje“ určující, zda je daná pozice aktivní.
\item Den (\texttt{DNY}) představuje číselníkovou tabulku pracovních dnů v týdnu s možností individuálního hodinového bonusu v daný den; přes asociativní tabulku \texttt{PRACOVNI\_POZICE\_A\_DNY} je poté možné určit dny ve které se pracovní pozice vykonává.
\end{list}

\subsection{Omezení a specifické požadavky}

\subsubsection{Integritní omezení}
\begin{list}{}{}
\item \texttt{CHECK}  „stopaz“ >= 0 zajistí, že délka filmu bude nezáporná.
\item \texttt{CHECK}  „kapacita“ >= 0 zajišťuje nezápornou teoretickou kapacitu prostoru pro diváky.
\item \texttt{CHECK}  „kapacita\_personalu“ >= 0 zajišťuje nezápornou teoretickou kapacitu prostoru pro personál.
\item \texttt{CHECK}  „cena“ >= 0 zajišťuje nezápornou cenu vstupenky.
\item \texttt{CHECK}  „mzda“ >= 0 zajišťuje nezáporný základ mzdy pro pracovní pozici.
\item \texttt{CHECK}  „sedadlo“ > 0 zajišťuje kladné číslo sedadla, jež odpovídá systému kina.
\item \texttt{UNIQUE}  „login“ znemožňuje duplicitní přihlašovací jména.
\end{list}

\subsubsection{Referenční integrita}
\begin{list}{}{}
\item Všechny cizí klíče jsou definovány jako \texttt{NOT} \texttt{NULL} (kromě „osoba\_id“ u vstupenky a~pracovní pozice, a~„misto\_id“ u pracovní pozice), což umožňuje evidovat anonymní či nezakoupenou vstupenku, \\neobsazenou pozici a pozici s prozatím nenaplánovaným místem výkonu práce.
\item Mazání nadřazených záznamů je omezeno existujícími vazbami -- např. nelze smazat záznam v tabulce \texttt{FILM}, na který existuje záznam s cizím klíčem v \texttt{PROMITANI}.
\end{list}

\subsubsection{Business pravidla implementovaná na aplikační úrovni}
\begin{list}{}{}
\item Počet vstupenek na promítání by v běžných případech neměl překročit kapacitu místa (\texttt{MISTO.kapacita}). Toto pravidlo není vynuceno na úrovni databáze -- kontroluje ho až aplikační vrstva.
\item Počet zaměstanců na určité pracovní pozici, v daný čas, a na daném místě by měl odpovídat kapacitě personálu daného místa (\texttt{MISTO.kapacita\_personalu}), aby pozice mohla fungovat. Toto omeznení bude kontrolovat až aplikační vrstva.
\item Promítání filmů by se neměla překrývat -- kontrolu časového překryvu (atributy „od“, „do“, „datum“ a „misto\_id“) řeší aplikační vrstva.
\end{list}

\subsubsection{Číselníkové a asociativní tabulky}
\begin{list}{}{}
\item \texttt{ZANR}, \texttt{DNY} a \texttt{SMENA} slouží jako číselníkové tabulky -- obsahují pouze statické údaje, do kterých běžný provoz pouze čte.
\item Asociativní tabulka \texttt{PRACOVNI\_POZICE\_A\_DNY} umožňuje M:N vazbu mezi pozicí a dny, kdy může být vykonávána -- bez dodatečných atributů.
\end{list}

\subsubsection{Technická omezení}
\begin{list}{}{}
\item Všechny časové údaje („od“, „do“ a „datum“) jsou uloženy jako \texttt{DATE} bez časové zóny -- tato aplikace zajišťuje konzistenci.
\item Maximální rozsah číselných sloupců (\texttt{NUMBER(38,2)}) odpovídá české měnové jednotce (Kč) s danou přesností na haléře.
\end{list}
Tato omezení zajišťují základní konzistenci dat a umožňují bezpečný provoz aplikace nad databází bez nutnosti dodatečných triggerů či procedur.

\newpage

% --- pohledy ---
\section{Reprezentativní databázové pohledy}

Tato kapitola obsahuje databázové pohledy, které poskytují přehledy a souhrnné informace o aktuálním stavu provozu kina.

\subsection{Pohled -- naklady-celkove.sql}

\subsubsection{Slovní popis dotazu}

Poskytuje celkový týdenní náklad kina na personál vyjádřený jedinou číselnou hodnotou. Výpočet probíhá ve dvou úrovních: nejprve se pro každého zaměstnance (\texttt{SOUCAST\_PERSONALU = 1}) vypočítá součet nákladů za všechny jeho směny v týdnu jako součty:
\begin{center}
$\bigl(\text{mzda na pozici} + \text{bonus za směnu} + \text{bonus za den}\bigr) \times 7$
\end{center}
kde číslem 7 je převedena 8hodinová směna včetně 1hodinové přestávky na skutečně odpracovaných 7 hodin. Tyto individuální sumy se pak agregují funkcí \texttt{SUM} přes všechny přihlašovací jména (atribut „login“), čímž vznikne jediná hodnota \texttt{NAKLADY\_NA\_TYDEN}. Pohled tak slouží jako rychlý ukazatel celkových osobních nákladů za týden.
\subsubsection{SQL kód dotazu}

\begin{lstlisting}
CREATE OR REPLACE VIEW NAKLADY_NA_TYDEN AS SELECT SUM(NAKLAD) AS NAKLADY_NA_TYDEN
FROM (
    SELECT LOGIN,
    	SUM(ZA_SMENU) AS NAKLAD
    FROM (
        SELECT o.LOGIN AS LOGIN,
        	((pp.MZDA + s.BONUS + d.BONUS)*7) AS ZA_SMENU
        FROM PRACOVNI_POZICE pp
        JOIN PRACOVNI_POZICE_A_DNY ppd ON ppd.PRACOVNI_POZICE_ID = pp.ID
        JOIN DNY d ON d.ID = ppd.DNY_ID
        JOIN SMENA s ON s.ID = pp.SMENA_ID
        JOIN OSOBA o ON o.ID = pp.OSOBA_ID 
        JOIN MISTO m ON m.ID = pp.MISTO_ID
        WHERE o.SOUCAST_PERSONALU = 1
        ORDER BY pp.NAZEV, o.LOGIN, s.NAZEV, m.NAZEV
    )
    GROUP BY LOGIN
);
\end{lstlisting}

\subsubsection{Odpověď na dotaz}

\begin{center}
\captionof{table}{Odpověď na naklady-celkove.sql}
\label{tab:naklady}
\vspace{5mm}
\begin{tabular}{c}
\toprule
\textbf{NAKLADY\_NA\_TYDEN} \\ \midrule
\csvreader[no head, late after line=\\, separator=semicolon]{csv/naklady-celkove.csv}{}%
{\csvcoli}%
\bottomrule
\end{tabular}
\end{center}

\newpage

\subsection{Pohled -- naklady-jednotlive.sql}

\subsubsection{Slovní popis dotazu}

Generuje detailní přehled týdenních nákladů rozpočítaný na jednotlivé zaměstnance. Pro každý záznam pracovní pozice a dne se nejprve vypočítá náklad za směnu stejným vzorcem:
\begin{center}
$\bigl(\text{mzda na pozici} + \text{bonus za směnu} + \text{bonus za den}\bigr) \times 7$
\end{center}
Hodnoty se následně seskupí podle \texttt{LOGIN} a sečtou, čímž vznikne dvojice \texttt{LOGIN} a \texttt{NAKLADY\_ZA\_TYDEN}. Díky tomu pro příklad manažer okamžitě vidí, kolik stojí provoz konkrétního člena personálu za celý týden.

\subsubsection{SQL kód dotazu}

\begin{lstlisting}
CREATE OR REPLACE VIEW NAKLADY_NA_TYDEN_JEDNOTLIVE AS SELECT LOGIN,
       SUM(ZA_SMENU) AS NAKLADY_ZA_TYDEN
FROM (
    SELECT o.LOGIN AS LOGIN,
           ((pp.MZDA + s.BONUS + d.BONUS)*7) AS ZA_SMENU
    FROM PRACOVNI_POZICE pp
    JOIN PRACOVNI_POZICE_A_DNY ppd ON ppd.PRACOVNI_POZICE_ID = pp.ID
    JOIN DNY d ON d.ID = ppd.DNY_ID
    JOIN SMENA s ON s.ID = pp.SMENA_ID
    JOIN OSOBA o ON o.ID = pp.OSOBA_ID 
    JOIN MISTO m ON m.ID = pp.MISTO_ID
    WHERE o.SOUCAST_PERSONALU = 1
    ORDER BY pp.NAZEV, o.LOGIN, s.NAZEV, m.NAZEV
)
GROUP BY LOGIN;
\end{lstlisting}

\subsubsection{Odpověď na dotaz}

\begin{center}
\captionof{table}{Odpověď na naklady-jednotlive.sql}
\label{tab:nakladyj}
\vspace{5mm}
\begin{tabular}{cc}
\toprule
\textbf{LOGIN} & \textbf{NAKLADY\_ZA\_TYDEN} \\ \midrule
\csvreader[no head, late after line=\\, separator=semicolon]{csv/naklady-jednotlive.csv}{}%
{\csvcoli & \csvcolii}%
\bottomrule
\end{tabular}
\end{center}

\newpage

\subsection{Pohled -- pracovni-pozice-info.sql}

\subsubsection{Slovní popis dotazu}

Podává úplnou mapu obsazených pracovních pozic včetně finančního ohodnocení. Každý řádek odpovídá unikátní kombinaci pozice, zaměstnanec, směna, den a místo. Sloupce \texttt{ZA\_HODINU} a \texttt{ZA\_SMENU} počítají přímé osobní náklady podle vzorců:
\begin{center}
$\text{mzda na pozici} + \text{bonus za směnu} + \text{bonus za den}$
\end{center}
\begin{center}
$\bigl(\text{mzda na pozici} + \text{bonus za směnu} + \text{bonus za den}\bigr) \times 7$
\end{center}
Příznak \texttt{FUNGUJE} převádí bitovou hodnotu \texttt{pp.FUNGUJE} na čitelné „ANO “ / „NE“. Filtr \texttt{SOUCAST\_PERSONALU~=~1} omezuje výstup pouze na aktivní zaměstnance; řazení podle názvu pozice, přihlašovacího jména, směny a místa usnadňuje kontrolu plánu i rozpočtu.

\subsubsection{SQL kód dotazu}

\begin{lstlisting}
CREATE OR REPLACE VIEW PRACOVNI_POZICE_INFO AS SELECT pp.NAZEV AS PRACOVNI_POZICE, 
       s.NAZEV AS SMENA, 
       o.LOGIN AS ZAMESTNANEC,
       m.NAZEV AS MISTO,
       d.DEN AS DEN,
       (pp.MZDA + s.BONUS + d.BONUS) AS ZA_HODINU,
       ((pp.MZDA + s.BONUS + d.BONUS)*7) AS ZA_SMENU,
       CASE pp.FUNGUJE
            WHEN 1 THEN 'ANO'
            WHEN 0 THEN 'NE'
       END AS FUNGUJE
FROM PRACOVNI_POZICE pp
JOIN PRACOVNI_POZICE_A_DNY ppd ON ppd.PRACOVNI_POZICE_ID = pp.ID
JOIN DNY d ON d.ID = ppd.DNY_ID
JOIN SMENA s ON s.ID = pp.SMENA_ID
JOIN OSOBA o ON o.ID = pp.OSOBA_ID 
JOIN MISTO m ON m.ID = pp.MISTO_ID
WHERE o.SOUCAST_PERSONALU = 1
ORDER BY pp.NAZEV, o.LOGIN, s.NAZEV, m.NAZEV;
\end{lstlisting}

\newpage

\subsubsection{Odpověď na dotaz}

\begin{center}
\captionof{table}{Odpověď na pracovni-pozice-info.sql (první část)}
\label{tab:pp1}
\vspace{5mm}
\scriptsize
\begin{tabular}{cccccccc}
\toprule
\textbf{PRACOVNI\_POZICE} & \textbf{SMENA} & \textbf{ZAMESTNANEC} & \textbf{MISTO} & \textbf{DEN} & \textbf{ZA\_HODINU} & \textbf{ZA\_SMENU} & \textbf{FUNGUJE} \\ \midrule
\csvreader[no head, late after line=\\, separator=semicolon]{csv/pracovni-pozice-info.csv}{}%
{\csvcoli & \csvcolii & \csvcoliii & \csvcoliv & \csvcolv & \csvcolvi & \csvcolvii & \csvcolviii}%
\bottomrule
\end{tabular}
\end{center}

\newpage

\begin{center}
\captionof{table}{Odpověď na pracovni-pozice-info.sql (druhá část)}
\label{tab:pp2}
\vspace{5mm}
\scriptsize
\begin{tabular}{cccccccc}
\toprule
\textbf{PRACOVNI\_POZICE} & \textbf{SMENA} & \textbf{ZAMESTNANEC} & \textbf{MISTO} & \textbf{DEN} & \textbf{ZA\_HODINU} & \textbf{ZA\_SMENU} & \textbf{FUNGUJE} \\ \midrule
\csvreader[no head, late after line=\\, separator=semicolon]{csv/pracovni-pozice-info-2.csv}{}%
{\csvcoli & \csvcolii & \csvcoliii & \csvcoliv & \csvcolv & \csvcolvi & \csvcolvii & \csvcolviii}%
\bottomrule
\end{tabular}
\end{center}

\newpage

\subsection{Pohled -- promitani-info.sql}

\subsubsection{Slovní popis dotazu}

Sestavuje kompletní statistiky každého promítání. Agregací přes \texttt{ID} promítání získá jeden řádek obsahující název filmu, formátované datum, název sálu, celkový počet vydaných vstupenek (včetně rezervací), počet skutečně zaplacených vstupenek a kapacitu sálu. Rozdíl mezi kapacitou a vydanými vstupenkami okamžitě prozradí obsazenost; poměr prodaných k vydaným ukazuje míru úspěšného prodeje. Pohled tedy slouží jako operativní dashboard pro kontrolu návštěvnické poptávky.

\subsubsection{SQL kód dotazu}

\begin{lstlisting}
CREATE OR REPLACE VIEW PROMITANI_INFO AS SELECT f.NAZEV, 
       TO_CHAR(p.DATUM, 'DD. MM. YYYY') AS DATUM,
       m.NAZEV AS MISTO,
       COUNT(v.ZAPLACENO) AS VYDANE_VSTUPENKY,
       COUNT(CASE WHEN v.ZAPLACENO = 1 THEN 1 END) AS PRODANE_VSTUPENKY,
       m.KAPACITA
FROM PROMITANI p
JOIN VSTUPENKA v ON v.PROMITANI_ID = p.ID
JOIN FILM f ON f.ID = p.FILM_ID
JOIN MISTO m ON m.ID = p.MISTO_ID
GROUP BY p.ID, f.NAZEV, p.DATUM, m.NAZEV, m.KAPACITA;
\end{lstlisting}

\subsubsection{Odpověď na dotaz}

\begin{center}
\captionof{table}{Odpověď na promitani-info.sql}
\label{tab:promitani}
\vspace{5mm}
\small
\begin{tabular}{cccccc}
\toprule
\textbf{NAZEV} & \textbf{DATUM} & \textbf{MISTO} & \textbf{VYDANE\_VSTUPENKY} & \textbf{PRODANE\_VSTUPENKY} & \textbf{KAPACITA} \\ \midrule
\csvreader[no head, late after line=\\, separator=semicolon]{csv/promitani-info.csv}{}%
{\csvcoli & \csvcolii & \csvcoliii & \csvcoliv & \csvcolv & \csvcolvi}%
\bottomrule
\end{tabular}
\end{center}

\subsection{Pohled -- smena-info.sql}

\subsubsection{Slovní popis dotazu}

Poskytuje čistý výpis všech směn uložených v tabulce \texttt{SMENA}. Vrací název směny, počáteční a koncový čas ve formátu \texttt{HH24:MI}. Protože neobsahuje žádné filtry ani agregační funkce, funguje jako referenční seznam pracovních úseků, které jsou dále využívány při plánování personálu i při finančních výpočtech v ostatních pohledech.

\subsubsection{SQL kód dotazu}

\begin{lstlisting}
CREATE OR REPLACE VIEW SMENA_INFO AS SELECT s.NAZEV AS NAZEV,
       TO_CHAR(s.OD, 'HH24:MI') AS ZACATEK,
       TO_CHAR(s.DO, 'HH24:MI') AS KONEC
FROM SMENA s;
\end{lstlisting}

\subsubsection{Odpověď na doataz}

\begin{center}
\captionof{table}{Odpověď na smena-info.sql}
\label{tab:smena}
\vspace{5mm}
\begin{tabular}{ccc}
\toprule
\textbf{NAZEV} & \textbf{ZACATEK} & \textbf{KONEC} \\ \midrule
\csvreader[no head, late after line=\\, separator=semicolon]{csv/smena-info.csv}{}%
{\csvcoli & \csvcolii & \csvcoliii}%
\bottomrule
\end{tabular}
\end{center}

\subsection{Pohled -- zisky-celkove.sql}

\subsubsection{Slovní popis dotazu}

Na úrovni celého kina sumuje dva klíčové ekonomické ukazatele:
\begin{list}{}{}
\item \texttt{PLANOVANE\_ZISKY} – potenciální tržbu, kdyby všechna místa byla prodána za průměrnou \newline cenu vstupenky
\item \texttt{DOSAVADNI\_ZISKY} – skutečné peněžní toky z již zaplacených vstupenek
\end{list}
Výpočet probíhá nejprve pro jednotlivá promítání:
\begin{center}
$\text{cena vstupenky} \times \text{kapacita místa}$
\end{center}
\begin{center}
$\text{cena vstupenky} \times \text{počet prodaných vstupenek}$
\end{center}
a následně se výsledky sečtou přes všechny promítání. Pohled tak dává okamžitou odpověď na otázku, \newline jak daleko je kino od maximálního možného výnosu.

\subsubsection{SQL kód dotazu}

\begin{lstlisting}
CREATE OR REPLACE VIEW ZISKY AS SELECT SUM(ODHAD_ZA_PROMITANI) AS PLANOVANE_ZISKY,
       SUM(ZATIM_ZA_PROMITANI) AS DOSAVADNI_ZISKY
FROM (
       SELECT f.NAZEV, 
              TO_CHAR(p.DATUM, 'DD. MM. YYYY') AS DATUM,
              (AVG(v.CENA) * m.KAPACITA) AS ODHAD_ZA_PROMITANI,
              (AVG(v.CENA) * COUNT(CASE WHEN v.ZAPLACENO = 1 THEN 1 END)) AS ZATIM_ZA_PROMITANI
       FROM PROMITANI p
       JOIN VSTUPENKA v ON v.PROMITANI_ID = p.ID
       JOIN FILM f ON f.ID = p.FILM_ID
       JOIN MISTO m ON m.ID = p.MISTO_ID
       GROUP BY p.ID, f.NAZEV, p.DATUM, m.KAPACITA
);
\end{lstlisting}

\subsubsection{Odpověď na doataz}

\begin{center}
\captionof{table}{Odpověď na zisky-celkove.sql}
\label{tab:zisky}
\vspace{5mm}
\begin{tabular}{cc}
\toprule
\textbf{PLANOVANE\_ZISKY} & \textbf{DOSAVADNI\_ZISKY} \\ \midrule
\csvreader[no head, late after line=\\, separator=semicolon]{csv/zisky.csv}{}%
{\csvcoli & \csvcolii}%
\bottomrule
\end{tabular}
\end{center}

\subsection{Pohled -- zisky-jednotlive.sql}

\subsubsection{Slovní popis dotazu}

Detailně rozkládá ekonomiku na úroveň jednotlivých promítání. Pro každý projekční blok vypočítá stejným vzorcem:
\begin{center}
$\text{cena vstupenky} \times \text{kapacita místa}$
\end{center}
\begin{center}
$\text{cena vstupenky} \times \text{počet prodaných vstupenek}$
\end{center}
Agregace \texttt{GROUP BY p.ID, f.NAZEV, p.DATUM, m.KAPACITA} zajistí, že každý řádek odpovídá právě jedné projekci. Výstup umožňuje sledovat, které konkrétní promítání zaostává či překonává očekávanou tržbu, a tedy kde je prostor pro optimalizaci cen nebo marketingu.

\subsubsection{SQL kód dotazu}

\begin{lstlisting}
CREATE OR REPLACE VIEW ZISKY_JEDNOTLIVE AS SELECT f.NAZEV, 
       TO_CHAR(p.DATUM, 'DD. MM. YYYY') AS DATUM,
       (AVG(v.CENA) * m.KAPACITA) AS ODHAD_ZA_PROMITANI,
       (AVG(v.CENA) * COUNT(CASE WHEN v.ZAPLACENO = 1 THEN 1 END)) AS ZATIM_ZA_PROMITANI
FROM PROMITANI p
JOIN VSTUPENKA v ON v.PROMITANI_ID = p.ID
JOIN FILM f ON f.ID = p.FILM_ID
JOIN MISTO m ON m.ID = p.MISTO_ID
GROUP BY p.ID, f.NAZEV, p.DATUM, m.KAPACITA;
\end{lstlisting}

\subsubsection{Odpověď na doataz}

\begin{center}
\captionof{table}{Odpověď na zisky-jednotlive.sql}
\label{tab:ziskyj}
\vspace{5mm}
\begin{tabular}{cccc}
\toprule
\textbf{NAZEV} & \textbf{DATUM} & \textbf{ODHAD\_ZA\_PROMITANI} & \textbf{ZATIM\_ZA\_PROMITANI} \\ \midrule
\csvreader[no head, late after line=\\, separator=semicolon]{csv/zisky-jednotlive.csv}{}%
{\csvcoli & \csvcolii & \csvcoliii & \csvcoliv}%
\bottomrule
\end{tabular}
\end{center}

\newpage

% --- scénáře ---
\section{Slovně komentované testovací scénáře}

V této kapitole jsou popsány testovací scénáře sloužící k ověření funkčnosti databázového modelu a pohledů v kontextu provozu kina.

\subsection{Scénář A}

Tento scénář simuluje kompletní interakci nového návštěvníka s (neexistující) webovou aplikací kina. Jeho cílem je ověřit, zda systém správně reaguje na běžný uživatelský průchod: zobrazení volných vstupenek, registraci, nákup, platbu a následné zobrazení vlastních vstupenek. Scénář je implementován v jednom SQL skriptu \texttt{a-vse.sql}, který postupně volá dílčí skripty (\texttt{a1.sql} až \texttt{a4.sql}). Každá část odpovídá jednomu kroku diváka v aplikaci a zároveň ověřuje konzistenci příslušných pohledů.

\subsubsection{Část -- Divák si prohlíží volná místa}

\begin{enumerate}
\item Divák hledá volné vstupenky na promítání a zobrazí se mu seznam všech volných míst pro všechna promítání.
\end{enumerate}

\begin{lstlisting}
SELECT f.NAZEV AS FILM,
       z.NAZEV AS ZANR,
       TO_CHAR(p.DATUM, 'DD. MM. YYYY') AS DATUM,
       TO_CHAR(p.OD, 'HH24:MI') AS ZACATEK,
       m.NAZEV AS MISTO,
       CASE
        WHEN v.SEDADLO IS NULL THEN 'Nečíslováno'
        ELSE TO_CHAR(v.SEDADLO) 
       END AS CISLO_SEDADLA,
       v.CENA AS CENA
FROM VSTUPENKA v
JOIN PROMITANI p ON p.ID = v.PROMITANI_ID
JOIN MISTO m ON m.ID = p.MISTO_ID
JOIN FILM f ON f.ID = p.FILM_ID
JOIN ZANR z ON z.ID = f.ZANR_ID
WHERE v.OSOBA_ID IS NULL;
\end{lstlisting}
\begin{list}{}{}
\item \texttt{PROMITANI\_INFO} zatím neobsahuje žádné změny -- počet \texttt{PRODANE\_VSTUPENKY} odpovídá stavu před nákupem.
\item Výsledek dotazu představuje „obrazovku“ s dostupnými sedadly, kterou vidí návštěvník.
\end{list}

\subsubsection{Část -- Registrace, dobití kreditu a nákup dvou vstupenek}

\begin{enumerate}
\item Divák vyplní formulář (přihlašovací jméno, e-mail, heslo) a registruje se.
\item Divák přejde na svůj profil a dobije svůj kredit na 400 Kč.
\item Vybere pondělní promítání filmu Leon, označí dvě volná místa a koupí dané vstupenky.
\end{enumerate}

\newpage

\begin{lstlisting}
INSERT INTO OSOBA (LOGIN, HESLO, EMAIL, KREDIT, SOUCAST_PERSONALU)
VALUES ('pan_leto', 'tajneheslo', 'pan_leto@seznam.cz', 0, 0);
COMMIT;

UPDATE OSOBA
SET KREDIT = KREDIT + 400
WHERE LOGIN LIKE 'pan_leto';
COMMIT;

UPDATE VSTUPENKA
SET OSOBA_ID = (SELECT ID FROM OSOBA WHERE LOGIN LIKE 'pan_leto'),
    ZAPLACENO = 1
WHERE ZAPLACENO = 0
AND OSOBA_ID IS NULL
AND PROMITANI_ID = 1
AND ROWNUM <= 2;
COMMIT;

UPDATE OSOBA
SET KREDIT = KREDIT - (
        (SELECT CENA FROM VSTUPENKA WHERE PROMITANI_ID = 1 AND ROWNUM <= 1)*2
    ) 
WHERE LOGIN LIKE 'pan_leto';
COMMIT;
\end{lstlisting}
\begin{list}{}{}
\item \texttt{PROMITANI\_INFO} -- sloupec \texttt{PRODANE\_VSTUPENKY} se okamžitě zvýší o 2.
\item \texttt{ZISKY} a \texttt{ZISKY\_JEDNOTLIVE} -- hodnota \texttt{ZATIM\_ZA\_PROMITANI} se navýší o cenu dvou vstupenek.
Žádné personální pohledy (\texttt{PRACOVNI\_POZICE\_INFO}, \texttt{NAKLADY\_*}) se nemění – operace se týká pouze návštěvníka.
\end{list}

\subsubsection{Část -- Návštěvník se přesvědčí, že jsou vstupenky opravdu zakoupil}

\begin{enumerate}
\item Divák zkontroluje volné vstupenky na promítání a zobrazí se mu seznam všech volných míst pro všechna promítání. Divák očekává, že už neuvidí vstupenky, které zakoupil.
\end{enumerate}

\newpage

\begin{lstlisting}
SELECT f.NAZEV AS FILM,
       z.NAZEV AS ZANR,
       TO_CHAR(p.DATUM, 'DD. MM. YYYY') AS DATUM,
       TO_CHAR(p.OD, 'HH24:MI') AS ZACATEK,
       m.NAZEV AS MISTO,
       CASE
        WHEN v.SEDADLO IS NULL THEN 'Nečíslováno'
        ELSE TO_CHAR(v.SEDADLO) 
       END AS CISLO_SEDADLA,
       v.CENA AS CENA
FROM VSTUPENKA v
JOIN PROMITANI p ON p.ID = v.PROMITANI_ID
JOIN MISTO m ON m.ID = p.MISTO_ID
JOIN FILM f ON f.ID = p.FILM_ID
JOIN ZANR z ON z.ID = f.ZANR_ID
WHERE v.OSOBA_ID IS NULL;
\end{lstlisting}

\begin{list}{}{}
\item Volné vstupenky -- po opětovném spuštění \texttt{a3.sql} (stejný dotaz jako \texttt{a1.sql}) chybí právě 2 řádky, které nyní mají nastaveno \texttt{OSOBA\_ID = <divákovo ID>}.
\item Výsledek dotazu představuje „obrazovku“ s dostupnými sedadly, kterou vidí návštěvník a která již neobsahuje vstupenky zakoupené divákem.
\end{list}

\subsubsection{Část -- Divák se přesvědčí, že vstupenky jsou jeho}

\begin{enumerate}
\item Divák na svém profilu zobrazí své vstupenky a zobrazí se mu seznam všech jeho vstupenek.
\end{enumerate}

\begin{lstlisting}
SELECT f.NAZEV AS FILM,
       z.NAZEV AS ZANR,
       TO_CHAR(p.DATUM, 'DD. MM. YYYY') AS DATUM,
       TO_CHAR(p.OD, 'HH24:MI') AS ZACATEK,
       m.NAZEV AS MISTO,
       CASE
        WHEN v.SEDADLO IS NULL THEN 'Nečíslováno'
        ELSE TO_CHAR(v.SEDADLO) 
       END AS CISLO_SEDADLA,
       v.CENA AS CENA
FROM VSTUPENKA v
JOIN PROMITANI p ON p.ID = v.PROMITANI_ID
JOIN MISTO m ON m.ID = p.MISTO_ID
JOIN FILM f ON f.ID = p.FILM_ID
JOIN ZANR z ON z.ID = f.ZANR_ID
WHERE v.OSOBA_ID = (SELECT ID FROM OSOBA WHERE LOGIN LIKE 'pan_leto');
\end{lstlisting}

\begin{list}{}{}
\item Výsledek dotazu představuje „obrazovku“ s divákovo vstupenkami – obsahuje přesně 2 řádky s filmem Leon, datem a časem pondělního promítání.
\end{list}

Skript \texttt{a-vse.sql} simuluje všechny tyto kroky scénáře v tomto pořadí.

\subsection{Scénář B}

Tento scénář simuluje interní proces přípravy a následného prodeje vstupenek na vybrané promítání z pohledu pokladního či administrátora kina. Cílem je ověřit, zda systém správně reflektuje vydání nových vstupenek, jejich částečný prodej a okamžité dopady na ekonomické přehledy. Celý průběh je implementován v jednom SQL skriptu \texttt{b-vse.sql}, který sekvenčně volá dílčí skripty \texttt{b1.sql} až \texttt{b5.sql}.

\subsubsection{Část -- Kontrola stavu před vydáním vstupenek}

\begin{enumerate}
\item Administrátor zkontroluje vstupenky k promítání číslo 3 a ověří, zda nejsou v systému již vydané.
\end{enumerate}

\begin{lstlisting}
SELECT f.NAZEV,
       TO_CHAR(p.OD, 'HH24:MI') AS ZACATEK_PROMITANI,
       TO_CHAR(p.DO, 'HH24:MI') AS KONEC_PROMITANI,
       TO_CHAR(p.DATUM, 'DD. MM. YYYY') AS DATUM,
       v.SEDADLO, 
       v.CENA,
       CASE v.ZAPLACENO
            WHEN 1 THEN 'ANO'
            WHEN 0 THEN 'NE'
       END AS ZAPLACENO,
       o.LOGIN
FROM VSTUPENKA v
JOIN PROMITANI p ON p.ID = v.PROMITANI_ID
JOIN FILM f ON f.ID = p.FILM_ID
LEFT JOIN OSOBA o ON o.ID = v.OSOBA_ID
WHERE p.ID = 3;
\end{lstlisting}

\begin{list}{}{}
\item Výsledek je prázdný -- potvrzuje, že před zahájením předprodeje neexistují žádné vstupenky k promítání číslo 3.
\item Pohled \texttt{PROMITANI\_INFO} pro dané promítání zobrazuje \texttt{VYDANE\_VSTUPENKY = 0} a \newline \texttt{PRODANE\_VSTUPENKY = 0}.
\end{list}

\subsubsection{Část -- Hromadné vydání vstupenek}

\begin{enumerate}
\item Administrátor se rozhodne vydat nové vstupenky a systém automaticky vygeneruje 30 číslovaných vstupenek (sedadla 1--30) s cenou 220 Kč dle zadání administrátora.
\end{enumerate}

\begin{lstlisting}
INSERT INTO VSTUPENKA (CENA, ZAPLACENO, SEDADLO, PROMITANI_ID)
VALUES (220, 0, 1, 3);

...

INSERT INTO VSTUPENKA (CENA, ZAPLACENO, SEDADLO, PROMITANI_ID)
VALUES (220, 0, 30, 3);
COMMIT;
\end{lstlisting}

\begin{list}{}{}
\item Pohled \texttt{PROMITANI\_INFO} se změní -- sloupec \texttt{VYDANE\_VSTUPENKY} naroste na 30.
\item Ekonomické pohledy \texttt{ZISKY} a \texttt{ZISKY\_JEDNOTLIVE} zvýší hodnotu \texttt{ODHAD\_ZA\_PROMITANI} o 
\begin{center}
	$30 \times 220$.
\end{center}
\end{list}

\subsubsection{Část -- Částečný prodej vstupenek zákazníkům}

\begin{enumerate}
\item V pokladně přijdou dva zákazníci: první (\texttt{ID} 2) zakoupí 5 vstupenek, druhý (\texttt{ID} 10) zakoupí 1 vstupenku. Pokladní označí vstupenky jako zaplacené a přiřadí je k účtům zákazníků.
\end{enumerate}

\begin{lstlisting}
UPDATE VSTUPENKA
SET ZAPLACENO = 1, OSOBA_ID = 2
WHERE ZAPLACENO = 0 AND PROMITANI_ID = 3
  AND OSOBA_ID IS NULL AND ROWNUM <= 5;
COMMIT;

UPDATE OSOBA
SET KREDIT = KREDIT - (220*5)
WHERE ID = 2;
COMMIT;

UPDATE VSTUPENKA
SET ZAPLACENO = 1, OSOBA_ID = 10
WHERE ZAPLACENO = 0 AND PROMITANI_ID = 3
  AND OSOBA_ID IS NULL AND ROWNUM <= 1;
COMMIT;

UPDATE OSOBA
SET KREDIT = KREDIT - 220
WHERE ID = 10;
COMMIT;
\end{lstlisting}

\begin{list}{}{}
\item Pohled \texttt{PROMITANI\_INFO} -- sloupec \texttt{PRODANE\_VSTUPENKY} naroste na 6.
\item Pohled \texttt{ZISKY} -- hodnota \texttt{ZATIM\_ZA\_PROMITANI} vzroste o 
\begin{center}
	$6 \times 220$
\end{center}
\item Osobní pohledy zůstávají beze změny (operace se týká pouze prodeje, nikoli personálu).
\end{list}

\subsubsection{Část -- Kontrola stavu po prodeji}

\begin{enumerate}
\item Administrátor znovu zkontroluje vstupenky k promítání číslo 3 a ověří, že 6 vstupenek je prodaných, zbylých 24 stále čeká na zákazníka.
\end{enumerate}

\begin{lstlisting}
SELECT f.NAZEV,
       TO_CHAR(p.OD, 'HH24:MI') AS ZACATEK_PROMITANI,
       TO_CHAR(p.DO, 'HH24:MI') AS KONEC_PROMITANI,
       TO_CHAR(p.DATUM, 'DD. MM. YYYY') AS DATUM,
       v.SEDADLO, 
       v.CENA,
       CASE v.ZAPLACENO
            WHEN 1 THEN 'ANO'
            WHEN 0 THEN 'NE'
       END AS ZAPLACENO,
       o.LOGIN
FROM VSTUPENKA v
JOIN PROMITANI p ON p.ID = v.PROMITANI_ID
JOIN FILM f ON f.ID = p.FILM_ID
LEFT JOIN OSOBA o ON o.ID = v.OSOBA_ID
WHERE p.ID = 3;
\end{lstlisting}

\begin{list}{}{}
\item Výsledek obsahuje 30 řádků: 6× \texttt{ZAPLACENO = 'ANO'} s přiřazeným loginem, \newline
24× \texttt{ZAPLACENO = 'NE'} a prázdný login.
\item Potvrzuje tak správnost předchozích kroků.
\end{list}

\subsubsection{Část -- Ekonomické shrnutí promítání}

\begin{enumerate}
\item Vedoucí kina si nechá zobrazit ekonomické shrnutí -- systém ukáže, kolik Kč již bylo utrženo a od kterých zákazníků.
\end{enumerate}

\begin{lstlisting}
SELECT CASE
        WHEN o.LOGIN IS NULL THEN '(Zatím neprodané)'
        ELSE o.LOGIN
       END AS LOGIN,
       SUM(v.CENA) AS ZISK_KINA,
       CASE 
        WHEN o.KREDIT IS NULL THEN ' '
        ELSE TO_CHAR (o.KREDIT) 
       END AS ZUSTATEK_PO_UTRATE
FROM VSTUPENKA v
JOIN PROMITANI p ON p.ID = v.PROMITANI_ID
LEFT JOIN OSOBA o ON o.ID = v.OSOBA_ID
WHERE p.ID = 3
GROUP BY v.OSOBA_ID, o.LOGIN, o.KREDIT;
\end{lstlisting}

\begin{list}{}{}
\item Výsledek obsahuje dva skutečné řádky (osob s \texttt{ID} 2 a 10) s částkami 1100 Kč a 220 Kč a zůstatky kreditů po nákupu.
\item Jeden řádek \texttt{'(Zatím neprodané)'} obsahuje částku 5 280 Kč ($24 \times 220$) -- plánovaný zisk z dosud neprodaných vstupenek.
\end{list}

Skript \texttt{b-vse.sql} simuluje všechny tyto kroky scénáře v tomto pořadí. 

\subsection{Scénář C}

Tento scénář demonstruje interní proces změny pracovní pozice zaměstnance v (neexistující) personální aplikaci kina. Jeho cílem je ověřit, zda systém správně zaznamená uvolnění původních pozic a přiřazení nové pozice jedinému zaměstnanci. Celý průběh je implementován v jednom SQL skriptu \texttt{c-vse.sql}, který postupně volá skripty \texttt{c1.sql} až \texttt{c3.sql}.

\subsubsection{Část -- Zobrazení aktuálně volných pracovních pozic}

\begin{enumerate}
\item Vedoucí personálního oddělení si zobrazí přehled všech pozic, které aktuálně nemají přiřazeného zaměstnance.
\end{enumerate}

\begin{lstlisting}
SELECT pp.NAZEV AS PRACOVNI_POZICE,
       s.NAZEV AS SMENA,
       CASE
        WHEN pp.OSOBA_ID IS NULL THEN 'Neurčen'
        ELSE TO_CHAR(pp.OSOBA_ID)
       END AS ZAMESTNANEC,
       CASE
        WHEN pp.MISTO_ID IS NULL THEN 'Neurčeno'
        ELSE TO_CHAR(m.NAZEV)
       END AS MISTO
FROM PRACOVNI_POZICE pp
JOIN SMENA s ON s.ID = pp.SMENA_ID
LEFT JOIN MISTO m ON m.ID = pp.MISTO_ID
WHERE pp.OSOBA_ID IS NULL;
\end{lstlisting}

\begin{list}{}{}
\item Výsledek dotazu představuje „obrazovku“ s volnými pracovními pozicemi, které jsou k dispozici pro nové nebo stávající zaměstnance.
\item Pohled \texttt{PRACOVNI\_POZICE\_INFO} zobrazuje pouze pozice s přiřazenými zaměstnanci, proto volné pozice v tomto pohledu chybí.
\end{list}

\subsubsection{Část -- Zaměstnanec si vybere novou pozici a uvolní staré}

\begin{enumerate}
\item Zaměstnanec s přihlašovacím jménem \texttt{vasek} projeví zájem o pozici „Noční hlídač“. Personální oddělení mu tuto pozici přiřadí a současně uvolní všechny jeho dosavadní pozice, protože nová role je časově náročná.
\end{enumerate}

\begin{lstlisting}
UPDATE PRACOVNI_POZICE
SET OSOBA_ID = (
    SELECT ID FROM OSOBA WHERE LOGIN LIKE 'vasek'
),
FUNGUJE = 1
WHERE NAZEV LIKE 'Noční hlídač';
COMMIT;

UPDATE PRACOVNI_POZICE
SET OSOBA_ID = NULL,
FUNGUJE = 0
WHERE NAZEV NOT LIKE 'Noční hlídač'
  AND OSOBA_ID = (
      SELECT ID FROM OSOBA WHERE LOGIN LIKE 'vasek'
);
COMMIT;
\end{lstlisting}

\begin{list}{}{}
\item Pohled \texttt{PRACOVNI\_POZICE\_INFO} se změní -- zaměstnanec \texttt{vasek} je nyní viditelný pouze u pozice „Noční hlídač“.
\item Pohledy \texttt{NAKLADY\_NA\_TYDEN} a \texttt{NAKLADY\_NA\_TYDEN\_JEDNOTLIVE} se automaticky přepočítají.
\item Ekonomické pohledy \texttt{ZISKY} a \texttt{ZISKY\_JEDNOTLIVE} se změní, protože „Noční hlídač“ je nově \newline vykonávaná a určitá jinačí už nikoliv.
\end{list}

\subsubsection{Část -- Kontrola aktuálně volných pozic po změně}

\begin{enumerate}
\item Vedoucí personálního oddělení znovu zobrazí seznam volných pozic a ověří, které pozice byly uvolněny a které nově obsazeny.
\end{enumerate}

\newpage

\begin{lstlisting}
SELECT pp.NAZEV AS PRACOVNI_POZICE,
       s.NAZEV AS SMENA,
       CASE
        WHEN pp.OSOBA_ID IS NULL THEN 'Neurčen'
        ELSE TO_CHAR(pp.OSOBA_ID)
       END AS ZAMESTNANEC,
       CASE
        WHEN pp.MISTO_ID IS NULL THEN 'Neurčeno'
        ELSE TO_CHAR(m.NAZEV)
       END AS MISTO
FROM PRACOVNI_POZICE pp
JOIN SMENA s ON s.ID = pp.SMENA_ID
LEFT JOIN MISTO m ON m.ID = pp.MISTO_ID
WHERE pp.OSOBA_ID IS NULL;
\end{lstlisting}

\begin{list}{}{}
\item Výsledek obsahuje všechny pozice, které byly dříve vykonávané zaměstnancem \texttt{vasek}, nyní označené jako 'Neurcen'.
\item Zároveň chybí pozice „Noční hlídač“, což potvrzuje, že byla úspěšně přiřazena.
\item Tento krok slouží jako kontrola správnosti předchozích SQL příkazů a zajišťuje konzistenci dat v personálním systému.
\end{list}

Skript \texttt{c-vse.sql} simuluje všechny tyto kroky scénáře v tomto pořadí. 

\newpage

% --- závěr ---
\section*{Závěr}

V rámci semestrální práce se mi podařilo vytvořit komplexní databázový model pro podporu provozu kina, který pokrývá klíčové oblasti -- od promítání filmů, prodeje vstupenek až po personální a ekonomické přehledy. Databáze je navržena v 3. normální formě, obsahuje vazby M:N, číselníkové tabulky a implementuje základní integritní a referenční omezení. Díky tomu je datový model konzistentní a připravený pro reálné nasazení, byť asi v omezeném provozu, protože by v produkčním prostředí pravděpodobně vyžadoval rozšíření o další funkce jako např. skladové hospodaření, věrnostní systém nebo účetní agendu.

Během realizace jsem si procvičil práci s relační databází Oracle, psaní komplexních SQL dotazů a vytváření pohledů pro agregaci dat v prostředí SQL Developer. Práce mě bavila zejména proto, že se týkala konkrétního a představitelného tématu. Mohl jsem při ní využít vlastní zkušenosti z návštěv kina a pracovat s reálnými daty o mých oblíbených filmech.

\end{document}
